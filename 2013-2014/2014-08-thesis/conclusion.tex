%!TEX root=paper/paper.tex
\chapter{Conclusion}\label{sec:conclusion}

\PM{Contribution}
We note a significant problem that has received little research attention: Anytime visual recognition.
The problem is motivated by the properties of human visual perception and by the need to effectively schedule computationally expensive state-of-the-art computer vision methods for different computational budgets.
We approach the problem from the perspective of reinforcement learning, and successfully learn fully general policies for selecting detector and classifier actions.
To evaluate our approaches, we introduce a new metric of Anytime performance, based on the area of performance vs. cost curve.
In all experiments, we show that having a dynamic state (allowing closed-loop policies) and planning ahead increases performance.

\PM{Detection}
We present a method for learning ``closed-loop'' policies for multi-class object detection, given existing object detectors and classifiers and a metric to optimize.
The method learns the optimal policy using reinforcement learning, by observing execution traces in training.
As always with reinforcement learning problems, defining the reward function requires some manual work.
Here, we derive it for the novel detection AP vs. Time evaluation that we suggest is useful for evaluating efficiency in recognition.
If detection on an image is cut off after only half the detectors have been run, our method does $66\%$ better than a random ordering, and $14\%$ better than an intelligent baseline.
In particular, our method learns to take action with no intermediate reward in order to improve the overall performance of the system.

\PM{Classification}
For the classification task, we additionally need to train classifiers for partially-observed sets of features.
We investigate methods such as different forms of imputation and classifier clustering for this task, and adjust the reward function and the featurization of the state.
We show improved Anytime performance on one synthetic classification task and two real visual recognition tasks using our method.
On a hierarchically-structured dataset, we additionally show that accuracy of predictions can be held constant for all budgets, while the specificty of predictions increases.

\PM{C-CNN}
We first consider approaches which can effectively reorder the sequence of regions to maximize the chance that correct detections will be found early, based on inference from relatively lightweight features.
We show that basic strategies, such as simply randomly reordering the boxes such that they do not have a degenerate spatial layout, provides a surprising boost, and that very simple features such as region and gradient statistics can effectively prioritize regions.
The main contribution, however, is the Cascaded CNN model, which adds a novel Reject layer to the AlexNet architecture and obtains an almost 10x speedup of the R-CNN detection method with only a 10\% degradation of state-of-the-art performance on PASCAL VOC detection.

\section{Future work}

\PM{Decisions}
Although computation devoted to scheduling actions is less significant than the computation due to running the actions in all of our work, our framework does not explicitly consider this decision-making cost.
A welcome extension would explicitly model this, perhaps drawing on eixsting theoretical work on meta-reasoning \parencite{Hay2012}.
This principle can also be applied to neural networks; we currently learn rejection thresholds in a process separate from the CNN training.
Future work should learn these thresholds jointly with the convolutional filters and the rest of the network layers.

\PM{Classification}
Nearest-neighbor method work really well for settings with partially observed sets of features, but were too slow in our evaluation.
Locality-sensitive hashign methods may provide an effective solution.
In particular, the method of \cite{Gao-NIPS-2011} could be valuably extended with hashing to maintain its model-free advantages at an acceptable speed.

\PM{Perception}
Beyond the aspects of practical deployment of vision systems that our work is motivated by, we are curious to further investigate our model as a tool to study human cognition and the time course of visual perception.
Only a few attempts have been made to explain this: for example, via sequential decision processes in \textcite{Hegde-Neuro-2008}.
While we have not made any claims about the biological mechanism of perception, our work in reinforcement learning-based feature selection as well as convolutional neural networks has explanatory potential if more tightly integrated in future work.
