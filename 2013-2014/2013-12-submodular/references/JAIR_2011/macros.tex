%
%
%

%
%
\newcommand{\term}[0]{adaptive\xspace}  %
                                %
\newcommand{\Term}[0]{Adaptive\xspace} %

%
%
\newcommand{\citecf}[1]{{\cite<{\cf}, >{#1}}}

%

\newcommand{\certifyingNS}{self--certifying}
\newcommand{\certifying}{self--certifying\xspace}
\newcommand{\Certifying}{Self--Certifying\xspace}

\newcommand{\probname}[0]{Stochastic Submodular Maximization} %

\newcommand{\smushname}[0]{compress\xspace} %
                                %
\newcommand{\AppendixA}[0]{The Appendix\xspace}
%
\newcommand{\appendixA}[0]{the Appendix\xspace}
%


%
%
%
%

%
%
\newcommand{\substitute}[3]{ \ensuremath{{#1}\!\left[{#2}/{#3}\right]} }
\newcommand{\condfun}[2]{\ensuremath{ {#1}^{#2} }}
\newcommand{\averaged}[1]{\ensuremath{#1_{\text{avg}}}} %
                                %
%



%
%
\newcommand{\entropy}[1]{\ensuremath{\mathbb{H}\paren{#1}}}
\newcommand{\infogain}[2]{\ensuremath{\mathbb{I}\paren{{#1} ; {#2}}}}
\newcommand{\support}{\op{support}}
\newcommand{\depth}{\operatorname{depth}} %
\newcommand{\perm}{\operatorname{perm}}   %
\renewcommand{\span}{\operatorname{span}}  
\newcommand{\indicator}[1]{\ensuremath{\one_{#1}}} 
\newcommand{\indicatorset}[1]{\ensuremath{\one_{\set{{#1}}}}} 

\newcommand{\event}[0]{{\mathcal{E}}} 
\newcommand{\instance}[0]{\ensuremath{\mathcal{I}}}

%


%
%
\newcommand {\costminsum}[1]{\ensuremath{c_{\Sigma}\!\paren{#1}}}
\newcommand {\costminsumsym}[0]{\ensuremath{c_{\Sigma}}}
\newcommand{\avgf}[0]{\averaged{f}} %
                                %
\newcommand{\prlzf}[1]{\ensuremath{ f_{\prlz}\!\paren{#1} }}
\newcommand{\prlzfsym}[1]{\ensuremath{ f_{\prlz_{{#1}}} }}
\newcommand{\avgg}[0]{\averaged{g}} %
                                %
\newcommand {\util}[2]{\ensuremath{u\!\paren{{#1}, {#2}}}}
%
%
\newcommand{\avgc}[1]{{c_{\text{avg}}{(#1)}}}
\newcommand{\cavg}[1]{{c_{\text{avg}}{(#1)}}}
\newcommand{\acst}[1]{{c_{\text{avg}}{(#1)}}}
\newcommand{\acstsym}[0]{{c_{\text{avg}}}}
\newcommand{\prlzcst}[2]{{c_{#1}{(#2)}}}
\newcommand{\wcst}[1]{{c_{\text{wc}}{(#1)}}}
\newcommand{\awstsym}[0]{{c_{\text{wc}}}}
\newcommand{\condcost}[2]{\ensuremath{c\paren{ {#2} \! \mid \! {#1} }  }}
%
%
\newcommand{\diff}[2]{\ensuremath{\Delta \hspace{-0.3mm}\paren{ {#2}\! \mid \! {#1} }  }
}  
%

%
%
\newcommand{\sequence}{\sigma}

\newcommand {\Tgreedy}{\ensuremath{T^{\text{greedy}}}}

\newcommand{\policy}[0]{\ensuremath{\pi}}
\newcommand{\greedypolicy}[0]{\ensuremath{\pi^{\text{greedy}}}}
\newcommand{\piavg}{\policy_{\text{avg}}^*}
\newcommand{\policycover}{\policy_{\text{avg}}^*}
\newcommand{\policyproof}{\policy_{\text{proof}}^*}

%
%
\newcommand{\strictprune}[2]{\ensuremath{  #1_{[\gets{#2}] }}} %
\newcommand{\laxprune}[2]{\ensuremath{  #1_{[{#2}\to]}}} %
\newcommand{\prune}[2]{\ensuremath{  #1_{[{#2}]} }  } %
\newcommand{\append}[2]{\ensuremath{{#1}@{#2}}} %

\newcommand{\underTree}[2]{\ensuremath{ #1^{-}_{[#2]}   }} %
\newcommand{\overTree}[2]{\ensuremath{ #1^{+}_{[#2]}   }} %
\newcommand{\treeCov}[3]{\ensuremath{ #1_{[#2] \cup \set{#3}}   }} %
\newcommand{\treeSmush}[3]{\ensuremath{ #1_{[#2] \cup \set{#3}}   }} %
\newcommand{\treeExtend}[3]{\ensuremath{ {#1}\!\# #2_{\set{#3}}   }} %
%

%
%
\newcommand{\node}[0]{\ensuremath{u}}
\newcommand{\elem}[0]{\ensuremath{e}}
\newcommand {\groundset}{\ensuremath{E}} %
\newcommand {\groundsubset}{\ensuremath{A}}%
                                %
\newcommand{\outcome}[0]{\ensuremath{o}} %
\newcommand{\outcomes}[0]{\ensuremath{O}} %
%
\newcommand{\mass}[1]{{\color{black}{\ensuremath{p\paren{#1}}}}}
\newcommand{\rlzmass}[1]{{\color{black}{\ensuremath{p\paren{#1}}}}}
\newcommand{\rlzmasssym}[0]{{\color{black}{\ensuremath{p}}}}
\newcommand{\rlzmassover}[2]{{\color{black}{\ensuremath{p_{#1}\!\paren{#2}}}}}
\newcommand{\rlzprior}[0]{{\color{black}{\ensuremath{p\paren{\rlz}}}}}
\newcommand{\rlz}[0]{{\color{black}{\ensuremath{\phi}}}} %
\newcommand{\rvrlz}[0]{{\color{black}{\ensuremath{\Phi}}}} %
\newcommand{\rlzt}[0]{\ensuremath{\rlz_{\text{true}}}} %
\newcommand{\rvprlz}[0]{{\color{black}{\ensuremath{\Psi}}}} %
\newcommand{\prlz}[0]{{\color{black}{\ensuremath{\psi}}}} %
\newcommand{\prlzsmall}[0]{\ensuremath{\psi}} %

\newcommand{\prlzsub}[2]{\substitute{\prlz}{#1}{#2}}
\newcommand{\prlzeo}[0]{\prlzsub{\elem}{\outcome}}
\newcommand{\prlzxo}[0]{\prlzsub{x}{\outcome}}
%
\newcommand{\vselim}[0]{\ensuremath{\vs_{\text{elim}}}} 

\newcommand{\rlzset}[0]{\mathcal{R}}
\newcommand{\prlzset}[0]{\mathcal{P}}
\newcommand{\htrue}[0]{\ensuremath{h_{\text{true}}}} %
                                %
\newcommand{\itrue}[0]{\ensuremath{i_{\text{true}}}} %
                                %
%



\newcommand{\model}[0]{\ensuremath{\mathcal{M}}}
\newcommand{\distrib}[0]{\ensuremath{\mathcal{D}}}

\newcommand{\quota}[0]{\ensuremath{Q}} %
\newcommand{\budget}[0]{\ensuremath{k}} %
\newcommand{\price}[0]{\ensuremath{\theta}} %
\newcommand{\charge}[1]{\ensuremath{C_{#1}}} %

\newcommand{\progress}[1]{\ensuremath{  \expct{f(\dom({#1}), \rvrlz) \mid
  \rvrlz \sim {#1}}  }}
\newcommand{\progressSet}[2]{\ensuremath{  \expct{f({#1}, \rvrlz) \mid
  \rvrlz \sim {#2}}  }}



%
%
\newcommand{\hypotheses}[0]{\ensuremath{H}} 
\newcommand{\data}[0]{\ensuremath{X}} 
\newcommand{\labels}[0]{\ensuremath{L}} 
\newcommand{\target}[0]{\ensuremath{h^*}} 
\newcommand{\error}[0]{\operatorname{error}} 
\newcommand{\prior}[0]{\ensuremath{p_{\hypotheses}}}
\newcommand{\vs}[0]{\ensuremath{V}\xspace} %
%
 \newcommand{\gbs}[0]{GBS\xspace} %
 \newcommand{\qcost}[2]{\ensuremath{q({#1}, {#2})}} %
\newcommand{\pcost}[2]{\ensuremath{\left\Vert{#1}\right\Vert_{#2}}} %

\newcommand{\ncost}[2]{\ensuremath{c({#1}, {#2})}} %
\newcommand{\vsmass}[2]{\ensuremath{\rho({#1}, {#2})}} %
\newcommand{\pmass}[3]{\ensuremath{\rho({#1}, {#2}; {#3})}} %
                                %
\newcommand{\timetosplit}[1]{\ensuremath{t({#1})}}
\newcommand{\vsmasssymbol}[0]{\ensuremath{\rho}} %
\newcommand{\played}[2]{\groundset({#1}, {#2})} 

\newcommand{\Tgbs}[0]{\ensuremath{T^{\text{GBS}}}} %
                                %
\newcommand{\massleft}[0]{\rho} 
%

\newcommand{\range}[0]{\operatorname{range}} 

\newcommand{\LHS}[1]{\operatorname{LHS}_{#1}}
\newcommand{\RHS}[1]{\operatorname{RHS}_{#1}}

%
%
\newcommand{\partset}[1]{\dot{#1}}
\newcommand{\itemparts}[0]{\partset{\groundset}}
%
%
\newcommand{\prt}[2]{{#1}\!\paren{{#2}}}
\newcommand{\iplayed}[2]{\itemparts({#1}, {#2})} 

%

%
%


\newcommand{\alive}[1]{{A({#1})}} 

%
\newcommand{\EMSE}{EMSE}
\newcommand{\MSE}{MSE}
\newcommand{\numsensors}{\ensuremath{n}}
\newcommand{\fEMSE}{\hat{f}_{\EMSE}}
\newcommand{\sensor}{\ensuremath{v}}
\newcommand{\sensors}{V}
\newcommand{\kernel}{K}
\newcommand{\pfail}[1]{p_{\text{fail}}\!\paren{#1} }

%
\newcommand{\numberof}[0]{\operatorname{rep}}
\newcommand{\invnumberof}[0]{\operatorname{rep}^{-1}}
\newcommand{\nmaps}{m}
\newcommand{\ntreasure}{t}
\newcommand{\nmatrix}{n}
\newcommand{\mapsize}{s}

%
\newcommand{\trace}{\tau}
\newcommand{\ptrace}[2]{\prlz\paren{{#1}, {#2}}}

\newcommand{\leaves}[0]{\mathcal{L}}
\newcommand{\parts}[0]{\mathcal{R}}
\newcommand{\pparts}[0]{\mathcal{P}}

\newcommand{\Xvec}{\mathbf{X}}

\newcommand{\gfloor}{\ensuremath{g^{\floor{k/2}}}}
 
\newcommand{\bound}{\beta}
\newcommand{\avgbound}{\beta_{\text{avg}}}
%

%
%
\newcommand{\daniel}[1]{\ifthenelse{\boolean{showcomments}}{\textcolor{red}{Daniel: #1}}{}}
\newcommand{\andreas}[1]{\ifthenelse{\boolean{showcomments}}{\textcolor{red}{Andreas: #1}}{}}

\newcommand{\myskip}{\myvspace{-0mm}}

\newcommand{\cf}{\emph{c.f.}\xspace}

\newcommand{\captionsmall}[1]{\caption{\small #1}}
\newcommand{\commentout}[1]{}

\newcommand{\family}{{\mathcal{F}}}
\newcommand{\sets}{\ensuremath{\mathcal{S}}}
\newcommand{\collection}{\ensuremath{\mathcal{C}}}

\newcommand{\cS}{{\mathcal{S}}}
\newcommand{\cA}{{\mathcal{A}}}
\newcommand{\cG}{{\mathcal{G}}}
\newcommand{\cE}{{\mathcal{E}}}
\newcommand{\cV}{{\mathcal{V}}}
\newcommand{\cB}{{\mathcal{B}}}
\newcommand{\cX}{{\mathcal{X}}}
\newcommand{\cY}{{\mathcal{Y}}}
\newcommand{\cH}{{\mathcal{H}}}
\newcommand{\cU}{{\mathcal{U}}}
\newcommand{\cW}{{\mathcal{W}}}
\newcommand{\cC}{{\mathcal{C}}}
\newcommand{\cO}{{\mathcal{O}}}

\newcommand{\scen}{{\mathcal{I}}}

\newcommand{\balpha}{{\mathbf{\alpha}}}
\newcommand{\ba}{{\mathbf{a}}}
\newcommand{\bh}{{\mathbf{h}}}
\newcommand{\bs}{{\mathbf{s}}}
\newcommand{\bt}{{\mathbf{t}}}
\newcommand{\bu}{{\mathbf{u}}}
\newcommand{\bv}{{\mathbf{v}}}
\newcommand{\bw}{{\mathbf{w}}}
\newcommand{\bx}{{\mathbf{x}}}
\newcommand{\by}{{\mathbf{y}}}
\newcommand{\bz}{{\mathbf{z}}}


%
%

%
\newlength{\presec}
\newlength{\postsec}
\newlength{\presubsec}
\newlength{\postsubsec}
\newlength{\prepara}
\newlength{\postpara}
\setlength{\presec}{-0mm}
\setlength{\postsec}{-0mm}
\setlength{\presubsec}{-0mm}
\setlength{\postsubsec}{-0mm}
\setlength{\prepara}{-0mm}
\setlength{\postpara}{-0mm}

\newenvironment{Enum}{
\vspace{-0.5em}
\begin{enumerate}
   \setlength{\itemsep}{0.25em}%
  \setlength{\parskip}{0em}
  \setlength{\parsep}{0em}}
{\end{enumerate}\vspace{-0.5em}}

\newcounter{myLISTctr}
\newcommand{\initOneLiners}{%
    \setlength{\itemsep}{0pt}
    \setlength{\parsep }{0pt}
    \setlength{\topsep }{0pt}
%
}
\newenvironment{OneLiners}[1][\ensuremath{\bullet}]
    {\begin{list}
        {#1}
        {\initOneLiners}}
    {\end{list}}


\newenvironment{proofof}[1]{
\noindent{\bf Proof of {#1}:}}
{\hfill$\blacksquare$
}

\newcommand{\ignore}[1]{}

\def \etal {et al.\ }
\def \argmax {\mathop{\rm arg\,max}}
\def \argmin {\mathop{\rm arg\,min}}

\newcommand{\ith}{\ensuremath{i^{\mathrm{th}}}\xspace}
\newcommand{\jth}{\ensuremath{j^{\mathrm{th}}}\xspace}
\newcommand{\op}[1]{\operatorname{#1}}
\newcommand{\one}{\mathbf{1}}
%
\newcommand{\poly}{\operatorname{poly}}
\newcommand{\paren} [1] {\ensuremath{ \left( {#1} \right) }}
\newcommand{\bigparen} [1] {\ensuremath{ \Big( {#1} \Big) }}
%
\renewcommand{\Pr}[1]{\ensuremath{\mathbb{P}\left[#1\right] }}
\newcommand{\prob}[1]{\ensuremath{\mathbb{P}\left[#1\right] }}
\newcommand{\probsym}[0]{\ensuremath{\mathbb{P}}}
\newcommand{\probover}[2]{\ensuremath{\mathbb{P}_{#1}\left[#2\right]
  }}
\newcommand{\proboverrlz}[2]{\ensuremath{\mathbb{P}\left[#2\right] }}
\newcommand{\E}[1]{\ensuremath{\mathbb{E}\left[#1\right] }}
%
\newcommand{\size}[1]{\ensuremath{\left|#1\right|}}
\newcommand{\ceil}[1]{\ensuremath{\left\lceil#1\right\rceil}}
\newcommand{\floor}[1]{\ensuremath{\left\lfloor#1\right\rfloor}}
%
\newcommand{\tuple}[1]{\ensuremath{\langle #1 \rangle}}
\newcommand{\func}[3]{\ensuremath{#1 : #2 \rightarrow #3}}
\newcommand{\problem} [1] {{\sc #1}}
\renewcommand{\implies}[0]{\ensuremath{\Rightarrow}}
%
\newcommand{\PositiveIntegers}{\ensuremath{\mathbb{Z^+}}}
\newcommand{\integers}{\ensuremath{\mathbb{Z}}}
\newcommand{\nats}{\ensuremath{\mathbb{N}}}
\newcommand{\reals}{\ensuremath{\mathbb{R}}}

%
\newcommand{\littleO}[1]{\ensuremath{o\paren{#1}}}
\newcommand{\bigO}[1]{\ensuremath{O\paren{#1}}}
\newcommand{\bigTheta}[1]{\ensuremath{\Theta\paren{#1}}}
\newcommand{\bigOmega}[1]{\ensuremath{\Omega\paren{#1}}}

%
\newcommand{\class} [1] {\textrm{#1}} %
\renewcommand{\P} {\class{P}}
\newcommand{\NP} {\class{NP}}

%

\newcommand{\compactdispmath}[1]{\[ \vspace{-1mm} {#1} \vspace{-0mm}  \]}


%
%
%
%
%
%
\newcommand{\roundindex}{\ensuremath{t}}
\newcommand{\partitionindex}{\ensuremath{k}}

\newcommand{\fullversion}{extended version\xspace}

\newcommand{\features}{\ensuremath{v}}
\newcommand{\NumFeatures}{\ensuremath{m}}
\newcommand{\NumRounds}{\ensuremath{T}}

\newcommand{\colorindex}{\ensuremath{c}}
\newcommand{\Color}{\ensuremath{c}}
\newcommand{\Colors}{\ensuremath{\bracket{\NumColors}}}
\newcommand{\Colorvec}{\ensuremath{ \vec{c} }}

%
\renewcommand{\c}{c}
%
%
%
%
%
%
%
%
%
%
%
%
%
%

%


\newcommand{\assignment}{S}  %
%
%

%


\newcommand {\groundsettimespartitions}{\ensuremath{\groundset\times\bracket{\NumPartitions}}} %
\newcommand {\matroidgroundset}{\ensuremath{\mathcal{X}}}
 %
\newcommand {\listgroundset} {\ensuremath{U}}
\newcommand{\OPT}{\textsf{OPT}}
\newcommand{\NULL}{\textsc{null}}

\newcommand{\obj}{\ensuremath{f}} %

\newcommand{\matroid}{\ensuremath{\mathcal{M}}}

%

\newcommand{\setoflist}[1]{\ensuremath{\operatorname{set}({#1})}}
\newcommand{\listofset}[1]{\ensuremath{\operatorname{list}({#1})}}
\newcommand {\ctr} {click-through-rate\xspace}
\newcommand {\ctrs} {click-through-rates\xspace}

%


%
\newcommand{\omitproof}[1]{}

\newcommand{\bracket}[1]{\left[#1\right]}
\newcommand{\Esub}[2]{\mathbb{E}_{#1}\bracket{#2}}
\newcommand{\sample}{\textsf{sample}}

\newcommand{\Alg}{\ensuremath{\mathcal{A}}}
\newcommand{\AllRows}{\ensuremath{\mathcal{R}_{\bracket{\J}}}}

\newcommand {\feasible} {\mathcal{F}}
\newcommand{\LocallyGreedy}{{\bf LocallyGreedy}\xspace}
\newcommand{\NonNegativeReals}{\ensuremath{\mathbb{R}_{\ge 0}}}
\newcommand{\NonNegativeIntegers}{\ensuremath{\mathbb{Z}_{\ge 0}}}

\newcommand{\NumColors}{\ensuremath{C}}
\newcommand{\NumPartitions}{\ensuremath{K}}
%
\newcommand{\OfflineGreedy}{{\sc TabularGreedy}\xspace}
\newcommand{\OnlineGreedy}{{\sc TGbandit}\xspace}
\newcommand{\Partition}{\ensuremath{P}}
\newcommand{\Regret}{\ensuremath{r}}
\newcommand{\Row}{\ensuremath{R}}
\newcommand{\Rows}{\ensuremath{\mathcal{R}}}
\newcommand{\SubGreedy}{{\bf SubGreedy}\ }



%
\vfuzz2pt %
\hfuzz2pt %
%
%
%
%
%
%
%
%
%
%
%
%
%

%
%
%
%

%
\DeclareMathOperator{\pr}{Pr} \DeclareMathOperator{\dist}{Dist}
\DeclareMathOperator{\dom}{dom}
\newcommand{\expct}[1]{\mathbb{E}\left[#1\right]}
\newcommand{\expctover}[2]{\mathbb{E}_{#1}\!\left[#2\right]}
\newcommand{\expctoverrlz}[2]{\mathbb{E}\left[#2\right]}

\newcommand{\norm}[1]{\left\Vert#1\right\Vert}
\newcommand{\abs}[1]{\left\vert#1\right\vert}
\newcommand{\set}[1]{\left\{#1\right\}}
\newcommand{\Real}{\mathbb R}
\newcommand{\eps}{\varepsilon}
\newcommand{\To}{\longrightarrow}
\newcommand{\BX}{\mathbf{B}(X)}
%
%
%
\DeclareMathOperator{\cP}{P} \DeclareMathOperator{\cNP}{NP}
\DeclareMathOperator{\DTIME}{DTIME}
\newcommand{\maxmarg}{{P}^{max}}

\newcommand{\rew}{\mathfrak{R}}

\newcommand{\hpts}{\hat{p}_{t,\bs}}
\newcommand{\hps}{\hat{p}_\bs}
\newcommand{\hp}{\hat{p}}


%

\newcommand{\denselist}{
    \itemsep -2pt\topsep-8pt\partopsep-8pt
}
\newcommand{\defref}[1]{Definition~\ref{#1}}
\newcommand{\tableref}[1]{Table~\ref{#1}}
\newcommand{\figref}[1]{Fig.~\ref{#1}}
\newcommand{\eqnref}[1]{Eq.~(\ref{#1})}
\newcommand{\secref}[1]{\S\ref{#1}}
\newcommand{\thmref}[1]{Theorem~\ref{#1}}
\newcommand{\corref}[1]{Corollary~\ref{#1}}
\newcommand{\propref}[1]{Proposition~\ref{#1}}
\newcommand{\lemref}[1]{Lemma~\ref{#1}}
\newcommand{\algref}[1]{Algorithm~\ref{#1}}


%
%
%

%
%

%
%
%
%
%
%
%
%
%
%


