%
\section{Introduction} \label{sec:intro}
%
%
\looseness -1 In many  problems arising in artificial intelligence one
needs to adaptively make a sequence of decisions, taking into account
observations about the outcomes of past decisions.  Often these
outcomes are uncertain, and one may only know a probability
distribution over them.  Finding optimal policies for decision making
in such partially observable stochastic optimization problems is
notoriously intractable (see, e.g., \citet{littman98computational}).  A fundamental challenge is to identify classes of planning problems for which simple solutions obtain (near-) optimal performance.


In this paper, we 
introduce the concept of \emph{\term submodularity}, and prove that
if a partially observable stochastic optimization problem satisfies this property, a simple adaptive greedy
algorithm is guaranteed to obtain near-optimal solutions. In fact, under reasonable complexity-theoretic assumptions, no polynomial time algorithm is able to obtain better solutions in general.
\Term
  submodularity generalizes the classical notion of submodularity\footnote{For
  an extensive treatment of submodularity, see the books
  of~\citet{fujishige05} and~\citet{SchrijverB}.}, which has
been successfully used to develop approximation algorithms for a
variety of non-adaptive optimization problems. 
 Submodularity, informally, is an intuitive notion of diminishing
 returns, which states that adding an element to a small set helps
 more than adding that same element to a larger (super-)\ set.  A
 celebrated result of the work of~\citet{nemhauser78} guarantees that for such
 submodular functions, a simple greedy algorithm, which adds the element that maximally increases the objective value, selects a near-optimal set of $k$ elements.  
Similarly, it is guaranteed to find a set of near-minimal cost that
achieves a desired quota of utility~\citep{wolsey82}, using 
near-minimum average time to do so~\citep{streeter08}.
Besides guaranteeing theoretical performance bounds, submodularity allows us
to speed up algorithms without loss of solution quality by using lazy evaluations \citep{minoux78},
often leading to performance improvements of several orders of
magnitude \citep{leskovec07}.  Submodularity has been shown to be very useful in a variety of problems in artificial intelligence \citep{krause09ijcaitut}.

The challenge in generalizing
submodularity to adaptive planning --- where the action taken in each
step depends on information obtained in the previous steps  --- is that feasible solutions are now
policies (decision trees or conditional plans) instead of subsets. We propose a natural
generalization of the diminishing returns property for adaptive
problems, which reduces to the classical characterization of submodular set
functions for deterministic distributions. We show how these results
of~\citet{nemhauser78},~\citet{wolsey82},~\citet{streeter08},
and~\citet{minoux78} generalize to the adaptive setting.  Hence,
we demonstrate how \term submodular optimization problems enjoy
theoretical and practical benefits similar to those of classical,
nonadaptive submodular problems. We further demonstrate the usefulness
and generality of the concept by showing how it captures known results
in stochastic optimization and active learning as special cases,
admits tighter performance bounds,  leads to natural generalizations
and allows us to solve new problems for which no performance guarantees were known.

As a first example, consider the problem of deploying (or controlling) a collection of
sensors to monitor some spatial phenomenon. Each sensor can cover a
region depending on its sensing range. Suppose we would like to find
the best subset of $k$ locations to place the sensors.  In this
application, intuitively, adding a sensor helps more if we have placed
few sensors so far and helps less if we have already placed many sensors. We can formalize this diminishing returns property using the notion of submodularity --
the total area covered by the sensors is a submodular function defined
over all sets of locations.  \citet{krause07nearoptimal} show that
many more realistic utility functions in sensor placement (such as the
improvement in prediction accuracy w.r.t.~some probabilistic model)
are submodular as well. Now consider the following stochastic variant:
Instead of deploying a fixed set of sensors, we deploy one sensor at a
time.  With a certain probability, deployed sensors can fail, and our
goal is to maximize the area covered by the functioning sensors.
Thus, when deploying the next sensor, we need to take into account
which of the sensors we deployed in the past failed.  This problem has
been studied by \citet{AsadpourNS08} for the case where each sensor
fails independently at random.  In this paper, we show that the
coverage objective is \term submodular, and use this concept to handle
much more general settings (where, e.g., rather than all-or-nothing
failures there are different types of sensor failures of varying severity).
%
We also consider a related problem where the goal is to place the minimum number of
sensors to achieve the maximum possible sensor coverage (i.e., the coverage
obtained by deploying sensors everywhere), or more generally the
goal may be to achieve any fixed percentage of the maximum possible
sensor coverage.  
Under the first goal, the problem is equivalent to one studied by~\citet{goemans06stochastic}, and 
generalizes a problem studied by \citet{liu08near}.  As with the
maximum coverage version, \term submodularity 
allows us to recover and generalize previous results.
 


As another example, consider a viral marketing problem, where we are
given a social network, and we want to influence as many people as
possible in the network to buy some product. We do that by giving the
product for free to a subset of the people, and hope that they
convince their friends to buy the product as well.  Formally, we have
a graph, and each edge $e$ is labeled by a number $0\leq p_e\leq 1$.  We ``influence'' a subset of nodes in the graph, and for each influenced node, their neighbors get randomly influenced according to the probability annotated on the edge connecting the nodes.  This process repeats until no further node gets influenced.  \citet{kempe03} show that the set function which quantifies the expected number of nodes influenced is submodular.  A natural stochastic variant of the problem is where we pick a node, get to see which nodes it influenced, then adaptively pick the next node based on these observations and so on.  We show that a large class of such adaptive influence maximization problems satisfies \term submodularity.


Our third application is in active learning, 
where we are given an unlabeled data set, and we would like to
adaptively pick a small set of examples whose labels imply all other
labels.  The same problem arises in automated diagnosis, where we have
hypotheses about the state of the system (e.g., what illness a patient has), and would like to perform tests to identify the correct hypothesis.
In both domains we want to pick examples / tests to shrink the remaining
version space (the set of consistent hypotheses) as quickly as
possible. Here, we show that the reduction in version space
probability mass is
\term submodular, and use that observation to prove that the adaptive
greedy algorithm is a near-optimal querying policy, recovering and 
%
generalizing  
results by \citet{kosaraju99} and~\citet{dasgupta04}. Our results for active learning and automated diagnosis are also related to recent results of \citet{guillory10interactive,guillory2011-noisy-interactive-submod-cover} who study generalizations of submodular set cover to an interactive setting.  In contrast to our approach however, \citeauthor{guillory10interactive} analyze worst-case costs, and use rather different technical definitions and proof techniques.


We summarize our main contributions below, and provide a more
technical summary in  Table~\ref{table:results}.
At a high level, our main contributions are: 
\begin{itemize}
\item We consider a particular class of partially observable adaptive stochastic optimization problems, which we prove to be hard to approximate in general.
\item We introduce the concept of \emph{\term submodularity}, and
  prove that if a problem instance satisfies this property, a simple
  adaptive greedy policy performs near-optimally, for both adaptive
  stochastic maximization and coverage, and also a natural min-sum objective.
\item We show how \term submodularity can be exploited by allowing the
  use of an accelerated adaptive greedy algorithm using lazy
  evaluations, and how we can obtain data-dependent bounds on the optimum.
\item We illustrate \term submodularity on several realistic problems,
  including Stochastic Maximum Coverage, Stochastic Submodular Coverage,
  Adaptive Viral Marketing, and 
  Active Learning.  For these applications, \term submodularity allows
  us to recover known results and prove natural generalizations.
\end{itemize}

%
%
\newcommand{\AS}{A.S.\xspace}
\newcommand{\AMS}{A.M.S.\xspace}

\begin{table}[t]
%
  \begin{tabular}{|p{1.5in}|p{3.4in}|p{0.8in}|}
\hline
\hspace{0.5in} {\bf Name} & \hspace{1.15in} {\bf New Results} &
\hspace{3mm} {\bf Location} \\ %
\hline
\AS Maximization & Tight $(1-1/e)$-approx.\ for \AMS objectives &
\secref{ssec:max-cover-objective}, page~\pageref{ssec:max-cover-objective}\\
\hline
\AS Min Cost Coverage & Tight logarithmic approx.\ for \AMS
objectives & \secref{sec:min-cost-cover}, page~\pageref{sec:min-cost-cover}\\
\hline
\AS Min Sum Cover & Tight $4$-approx.\ for \AMS objectives &
\secref{sec:min-sum-cover}, page~\pageref{sec:min-sum-cover}\\
\hline
Data Dependent Bounds & Generalization to
\AMS functions & \secref{ssec:max-cover-objective}, page~\pageref{ssec:max-cover-objective}\\
\hline
Accelerated Greedy & Generalization of lazy evaluations to the
adaptive setting & \secref{sec:the-greedy-policy}, page~\pageref{sec:the-greedy-policy}\\
\hline    
Stochastic Submodular Maximization &  Generalization of the previous $(1-1/e)$-approx.\ to \mbox{arbitrary} 
per--item set distributions, and to item costs &
\secref{sec:stochastic-maximization}, page~\pageref{sec:stochastic-maximization}\\
\hline    
Stochastic Set Cover & Generalization of the previous
$(\ln(n)+1)$-approx.\ to \mbox{arbitrary} per-item set distributions, with
item costs & \secref{sec:stochastic-set-cover}, page~\pageref{sec:stochastic-set-cover}\\
\hline    
\mbox{Adaptive Viral} \mbox{Marketing} & Adaptive analog of previous
$(1-1/e)$-approx.\ for non-adaptive viral marketing, under more
general reward functions; tight logarithmic approx.\ for the adaptive
min cost cover version & \secref{sec:viral-marketing}, page~\pageref{sec:viral-marketing}\\
\hline    
Active Learning & Improved approx.\ factor of generalized binary search and its approximate versions
%
with and without item costs & \secref{sec:active-learning}, page~\pageref{sec:active-learning}\\
\hline    
%
%
Hardness in the absence of Adapt.\ Submodularity &
$\Omega(|\groundset|^{1-\epsilon})$-approximation hardness for \AS
Maximization, Min Cost Coverage, and Min-Sum Cover, if $f$ is not
\term submodular.  & \secref{sec:hardness}, page~\pageref{sec:hardness}\\
\hline    
 \end{tabular}
  \caption{Summary of our theoretical results.  \AS is shorthand for
    ``adaptive stochastic'', and \AMS is shorthand for ``adaptive
    monotone submodular.'' \vspace{-5mm}}  \label{table:results}
\end{table}


%
%
%
%


\subsection{Organization} %

\looseness -1 This article is organized as follows.
In \secref{sec:problem-statment} (page~\pageref{sec:problem-statment}) we set up notation and formally define
the relevant adaptive optimization problems for general objective
functions. 
\emph{For the reader's convenience, we have also provided a reference table of
important symbols on page~\pageref{table:symbol-table}.}
In~\secref{sec:term-submodularity} (page~\pageref{sec:term-submodularity}) we review the classical notion of submodularity and introduce the novel 
\term submodularity property.  In~\secref{sec:the-greedy-policy}
(page~\pageref{sec:the-greedy-policy}) we introduce
the adaptive greedy policy, as well as an accelerated variant.
In~\secref{sec:greedy} (page~\pageref{sec:greedy}) we discuss the theoretical guarantees
that the adaptive greedy policy enjoys when applied to 
problems with \term submodular objectives.
Sections~\ref{sec:stochastic-maximization} through~\ref{sec:active-learning}
provide examples on how to apply the \term submodular framework to
various applications, namely Stochastic Submodular Maximization
(\secref{sec:stochastic-maximization}, page~\pageref{sec:stochastic-maximization}), Stochastic Submodular Coverage
(\secref{sec:stochastic-set-cover}, page~\pageref{sec:stochastic-set-cover}), 
Adaptive Viral Marketing (\secref{sec:viral-marketing}, page~\pageref{sec:viral-marketing}), and 
Active Learning (\secref{sec:active-learning}, page~\pageref{sec:active-learning}).
In~\secref{sec:experiments} (page~\pageref{sec:experiments}) we report empirical results on two sensor
selection problems.
In~\secref{sec:adaptgap} (page~\pageref{sec:adaptgap}) we discuss the adaptivity gap of the problems we
consider, and in~\secref{sec:hardness} (page~\pageref{sec:hardness}) we prove hardness results
indicating that problems which are not \term submodular  can be extremely
inapproximable under reasonable complexity assumptions.
We review related work in~\secref{sec:related-work} (page~\pageref{sec:related-work}) and provide concluding remarks in
\secref{sec:conclusions} (page~\pageref{sec:conclusions}).  \AppendixA (page~\pageref{sec:proofs}) gives details of how to incorporate item
costs and includes all of the proofs omitted from the main text. 



%
%
%
%
%
